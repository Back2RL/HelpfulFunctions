\documentclass[12pt]{scrartcl}

\usepackage{ucs}
\usepackage[utf8x]{inputenc}
\usepackage[T1]{fontenc}
\usepackage{amsmath,amssymb,amstext} %Mathe
\usepackage{graphicx} %zum Bilder einfügen
\usepackage[ngerman]{babel}
\usepackage[automark]{scrpage2}	%für Kopf/Fußzeilen
\pagestyle{scrheadings}
\clearscrheadfoot
\ifoot[]{\author}
\ofoot[]{\pagemark}

\title{Hier Titel einfügen}
\author{L. O.}
\date{\today{}...}


\begin{document}
\maketitle
\tableofcontents 	%Inhaltsverzeichnis
\newpage	%neue Seite anfangen
Hier steht normaler Text
\begin{center}
Zentriert
\end{center}
\section{1st section}	%Sektion 1. Inhaltsverzeichniseintrag
%\section, \subsection, \subsubsection, \paragraph und \subparagraph zur Verfügung. In anderen Dokumentklassen gibt es z.B. noch \chapter und \part 
\section{2nd section} %Sektion 1. Inhaltsverzeichniseintrag
\label{sec:2nd-section} %Möcglichkeit zum Verweisen

\begin{itemize} %unterpunkte/Themen
\item item
\begin{itemize}
\item item
\end{itemize}
\item item
\begin{enumerate}	%Aufzählung
\item enumerate
\end{enumerate}
\end{itemize}

\section{3rd section}
\label{sec:3rd-section}

\subsection{Unterstufe}
\label{sec:unterstufe}

\subsection{Oberstufe} 
\label{sec:oberstufe}

\begin{equation*}
a + 2 = c
\end{equation*}

\begin{equation*}
a_{ij} - a_2 = 0
\end{equation*}

\begin{equation*}
\frac{1}{a} + \frac{1}{b} = \frac{a+b}{ab}
\end{equation*}

\begin{equation*}
\sigma + \tau = \alpha
\end{equation*}

\begin{equation}
\label{eq:1}
\left( \frac{a}{b} \right)' = \frac{a'b-ab'}{b^{2}}
\end{equation} 
%So funktioniert das mit allen Klammern: ()[] bzw. \{ und \}. Da die geschwungen Klammern für LaTeX eine interne Bedeutung haben, müssen sie mit vorangestelltem \ angeschrieben werden, wenn man sie als Zeichen verwenden möchte. 

Es gilt die Invariante $b \neq 0$.

\begin{equation}
\label{eq:2}
\int\limits_{a}^{b} x^{2} \, dx = \frac{ b^{3} - a^{3} }{3}
\end{equation}

\begin{equation}
\label{eq:3}
c = \sqrt{ a^{2} + b^{2} }
\end{equation}

\section{Fortgeschrittene Anwendung}
\label{sec:fortg-anwend}

\subsection{subsection}
\label{sec:subsection}

In Abschnitt \ref{sec:1st-section} wurde ein Mädchen namens
Alice erwähnt. Was sie im Wunderland erlebt, kann in einem Buch
nachgelesen werden.

\subsection{Analyse}
\label{sec:analyse}

Die Gleichungen \eqref{eq:1} bis \eqref{eq:3} beherrschen wir bestens.
Alice, von der wir auf Seite \pageref{sec:einleitende-worte} gehört
haben, kennt diese Gleichungen wahrscheinlich nicht.

Die Gleichungen \eqref{eq:1} bis \eqref{eq:3} beherrschen wir bestens.
Alice, von der wir auf Seite \pageref{sec:einleitende-worte} gehört
haben, kennt diese Gleichungen wahrscheinlich nicht.

%\begin{center}
%\includegraphics[width=0.1\textwidth]{TUG-logo}
%\end{center}

Das Bild zeigt unser Logo\footnote{Bitte korrekt verwenden.}.

Das Bild zeigt unser Logo nochmal\footnote{Bitte auch korrekt verwenden.}.

\end{document}
